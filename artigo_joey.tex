%% abtex2-modelo-artigo.tex, v-1.9.6 laurocesar
%% Copyright 2012-2016 by abnTeX2 group at http://www.abntex.net.br/ 
%%
%% This work may be distributed and/or modified under the
%% conditions of the LaTeX Project Public License, either version 1.3
%% of this license or (at your option) any later version.
%% The latest version of this license is in
%%   http://www.latex-project.org/lppl.txt
%% and version 1.3 or later is part of all distributions of LaTeX
%% version 2005/12/01 or later.
%%
%% This work has the LPPL maintenance status `maintained'.
%% 
%% The Current Maintainer of this work is the abnTeX2 team, led
%% by Lauro César Araujo. Further information are available on 
%% http://www.abntex.net.br/
%%
%% This work consists of the files abntex2-modelo-artigo.tex and
%% abntex2-modelo-references.bib
%%

% ------------------------------------------------------------------------
% ------------------------------------------------------------------------
% abnTeX2: Modelo de Artigo Acadêmico em conformidade com
% ABNT NBR 6022:2003: Informação e documentação - Artigo em publicação 
% periódica científica impressa - Apresentação
% ------------------------------------------------------------------------
% ------------------------------------------------------------------------

\documentclass[
	% -- opções da classe memoir --
	article,			% indica que é um artigo acadêmico
	12pt,				% tamanho da fonte
	oneside,			% para impressão apenas no recto. Oposto a twoside
	a4paper,			% tamanho do papel. 
	% -- opções da classe abntex2 --
	%chapter=TITLE,		% títulos de capítulos convertidos em letras maiúsculas
	%section=TITLE,		% títulos de seções convertidos em letras maiúsculas
	%subsection=TITLE,	% títulos de subseções convertidos em letras maiúsculas
	%subsubsection=TITLE % títulos de subsubseções convertidos em letras maiúsculas
	% -- opções do pacote babel --
	english,			% idioma adicional para hifenização
	brazil,				% o último idioma é o principal do documento
	sumario=tradicional
	]{abntex2}


% ---
% PACOTES
% ---

% ---
% Pacotes fundamentais 
% ---
\usepackage{lmodern}			% Usa a fonte Latin Modern
\usepackage[T1]{fontenc}		% Selecao de codigos de fonte.
\usepackage[utf8]{inputenc}		% Codificacao do documento (conversão automática dos acentos)
\usepackage{indentfirst}		% Indenta o primeiro parágrafo de cada seção.
\usepackage{nomencl} 			% Lista de simbolos
\usepackage{color}				% Controle das cores
\usepackage{graphicx}			% Inclusão de gráficos
\usepackage{microtype} 			% para melhorias de justificação
% ---
		
% ---
% Pacotes adicionais, usados apenas no âmbito do Modelo Canônico do abnteX2
% ---
\usepackage{lipsum}				% para geração de dummy text
% ---
		
% ---
% Pacotes de citações
% ---
\usepackage[brazilian,hyperpageref]{backref}	 % Paginas com as citações na bibl
\usepackage[alf]{abntex2cite}	% Citações padrão ABNT
% ---

% ---
% Configurações do pacote backref
% Usado sem a opção hyperpageref de backref
\renewcommand{\backrefpagesname}{Citado na(s) página(s):~}
% Texto padrão antes do número das páginas
\renewcommand{\backref}{}
% Define os textos da citação
\renewcommand*{\backrefalt}[4]{
	\ifcase #1 %
		Nenhuma citação no texto.%
	\or
		Citado na página #2.%
	\else
		Citado #1 vezes nas páginas #2.%
	\fi}%
% ---

% ---
% Informações de dados para CAPA e FOLHA DE ROSTO
% ---
\titulo{Suíte de testes\\para biblioteca de engenharia}
\autor{Jonhnatha Trigueiro}
\local{Brasil}
\data{2016}
% ---

% ---
% Configurações de aparência do PDF final

% alterando o aspecto da cor azul
\definecolor{blue}{RGB}{41,5,195}

% informações do PDF
\makeatletter
\hypersetup{
     	%pagebackref=true,
		pdftitle={\@title}, 
		pdfauthor={\@author},
    	pdfsubject={Modelo de artigo científico com abnTeX2},
	    pdfcreator={LaTeX with abnTeX2},
		pdfkeywords={abnt}{latex}{abntex}{abntex2}{atigo científico}, 
		colorlinks=true,       		% false: boxed links; true: colored links
    	linkcolor=blue,          	% color of internal links
    	citecolor=blue,        		% color of links to bibliography
    	filecolor=magenta,      		% color of file links
		urlcolor=blue,
		bookmarksdepth=4
}
\makeatother
% --- 

% ---
% compila o indice
% ---
\makeindex
% ---

% ---
% Altera as margens padrões
% ---
\setlrmarginsandblock{3cm}{3cm}{*}
\setulmarginsandblock{3cm}{3cm}{*}
\checkandfixthelayout
% ---

% --- 
% Espaçamentos entre linhas e parágrafos 
% --- 

% O tamanho do parágrafo é dado por:
\setlength{\parindent}{1.3cm}

% Controle do espaçamento entre um parágrafo e outro:
\setlength{\parskip}{0.2cm}  % tente também \onelineskip

% Espaçamento simples
\SingleSpacing

% ----
% Início do documento
% ----
\begin{document}

% Seleciona o idioma do documento (conforme pacotes do babel)
%\selectlanguage{english}
\selectlanguage{brazil}

% Retira espaço extra obsoleto entre as frases.
\frenchspacing 

% ----------------------------------------------------------
% ELEMENTOS PRÉ-TEXTUAIS
% ----------------------------------------------------------

%---
%
% Se desejar escrever o artigo em duas colunas, descomente a linha abaixo
% e a linha com o texto ``FIM DE ARTIGO EM DUAS COLUNAS''.
% \twocolumn[    		% INICIO DE ARTIGO EM DUAS COLUNAS
%
%---
% página de titulo
\maketitle

% resumo em português
\begin{resumoumacoluna}
A proposta do artigo é criar uma suite de testes para biblioteca de engenharia. Esta abordará a construção da biblioteca e os passos necessários para torná-la testável. 
Haverá a configuração do sistema de build para que esta possa ser integrada às ferramentas que realizem os testes e métricas de forma automatizada, assim como rege o funcionamento da Integração contínua.
Também abordaremos as forma de como a comunidade opensource conseguirá manipular o código, criando um modelo de trabalho progressivo.

 \vspace{\onelineskip}
 
 \noindent
 \textbf{Palavras-chave}: teste, suite de testes, Integração Contínua
\end{resumoumacoluna}

% ]  				% FIM DE ARTIGO EM DUAS COLUNAS
% ---

% ----------------------------------------------------------
% ELEMENTOS TEXTUAIS
% ----------------------------------------------------------
\textual

% ----------------------------------------------------------
% Introdução
% ----------------------------------------------------------
\section*{Introdução}
\addcontentsline{toc}{section}{Introdução}

Desenvolvimento guiado à testes é um processo de criação de software
que se baseia na repetição de um ciclo de desenvolvimento muito curto e definido.
Como objetivos dessa abordagem se pode citar os seguintes: Produtividade no desenvolvimento,
código fonte de fácil manutenção e confiabilidade do código.

A integração contínua, na prática, é uma técnica de mesclar o código de vários desenvolvedores em uma linha comum compartilhada de forma a ser feito várias vezes ao dia. Foi proposto inicialmente por Grady Booch em seu método 1991 e é usado com muita intensidade no método de desenvolvimento ágil XP, que advoga realizar o processo de integração continua mais de uma vez por dia.

% ----------------------------------------------------------
% Seção de explicações
% ----------------------------------------------------------i

\section{Desenvolvimento}

\subsection{Teste}

O conceito de teste é um termo abrangente. Vias gerais, significa firmar convicção de um juiz sobre a verdade dos fatos alegados pelas partes em juízo. Em programação este termo pode ser lido como "a convicção do desenvolvedor" sobre a confiabilidade de que o sistema desenvolvido irá funcionar conforme esperado.

Há algumas fases para o teste de software. Cada um visa abranger um escopo no ciclo de desenvolvimento. Podemos citar os seguintes:
\begin{itemize}
	\item Teste de unidade;
	\item Teste de integração;
	\item Teste de sistema;
	\item Teste de aceitação e
	\item Teste de operação.
\end{itemize}

A rotina de testes, apesar de ser de fácil compreensão, requer disciplina por parte do desenvolvedor, que, precisa pensar em código fracamente acoplado. Leia-se "código fracamente acoplado" aquele o qual os as funcionalidades são dispostas em funções, com entradas e saídas simpels, de modo que se comprometa a realizar apenas uma competência.

\subsubsection{Teste de unidade}

É o menor grupo de testes. Estes testam funções específicas. Dado uma entrada e saída conhecidas pelo desenvolvedor, a suite de testes comparará o resultado da aplicação da função com a saída esperada. Dependendo do método de asserção escolhido, o teste irá passar ou falhar.

\subsection{Biblioteca de Engenharia}
O software o qual será desenvolvido a suite de testes será a biblioteca de Engenharia, criada colaborativamente pelo professor André Felipe Dantas.\\
A biblioteca foi desenvolvida em C++ e possui as seguintes funcionalidades, inicialmente:

\begin{itemize}
	\item Trabalhar matrizes;
	\item Trabalhar com polinômios;
	\item Realizar operações matriciais;
\end{itemize}

Também é previsto para a biblioteca, como roadmap os seguintes recursos:

\begin{itemize}
	\item Representar modelos matemáticos para sistemas conhecidos;
	\item Sintonizar sistemas dado modelo escolhido, método de sintonia e entrada de dados, achando melhor combinação;
\end{itemize}

Usaremos os seguintes recursos:

\subsection{GIT}
O GIT é um software de versionamento de código, desenvolvido por Linux Torvalds, que faz com que o código fonte possa ser mantido sobre um sistema de controle de versão. Este é importante pois contribui para o processo colaborativo, essencial na comunidade Opensource.

\subsection{TravisCI}
É um serviço se integração contínua usado para executar o processo de build e teste de projetos de software geralmente hospedados no GitHub.

\subsection{CXXTest}
É um framework de testes unitários para C++ que é similar, em essência, ao JUnit, CppUnit e xUnit. É de fácil utilização porque não requer precompilar a biblioteca de teste. Também não requer recursos avançados, como RTTI por exemplo, e possui uma forma bastante flexível de descoberta de testes.

\subsection{Processo de desenvolvimento}
Será mapeado, individualmente os módulos da aplicação fazendo uma análise do quão acoplado está o código. Com base nesses dados será iniciado o processo de desacoplação de código e criação dos testes sob demanda.

Os testes precisarão, necessariamente começar pelos métodos construtores e ir avançando em cada método subsequente. Também é necessário fazer asserções simples.

Ao passo que irá sendo construído a suite, o mesmo terá seu versionamento no GitHub e iniciado um projeto de build automatizado no Travis CI.

% ---
% Finaliza a parte no bookmark do PDF, para que se inicie o bookmark na raiz
% ---
\bookmarksetup{startatroot}% 
% ---

% ---
% Conclusão
% ---
%\section*{Considerações finais}
%\addcontentsline{toc}{section}{Considerações finais}

%\lipsum[1]

%\begin{citacao}
%\lipsum[2]
%\end{citacao}

%\lipsum[3]

% ----------------------------------------------------------
% ELEMENTOS PÓS-TEXTUAIS
% ----------------------------------------------------------
\postextual

% ---
% Título e resumo em língua estrangeira
% ---

% \twocolumn[    		% INICIO DE ARTIGO EM DUAS COLUNAS

% titulo em inglês
%\titulo{Canonical academic article model with \abnTeX}
%\emptythanks
%\maketitle

% resumo em português
% ]  				% FIM DE ARTIGO EM DUAS COLUNAS
% ---

% ----------------------------------------------------------
% Referências bibliográficas
% ----------------------------------------------------------
\bibliography{abntex2-modelo-references}

% ----------------------------------------------------------
% Glossário
% ----------------------------------------------------------
%
% Há diversas soluções prontas para glossário em LaTeX. 
% Consulte o manual do abnTeX2 para obter sugestões.
%
%\glossary

% ----------------------------------------------------------
% Apêndices
% ----------------------------------------------------------

% ---
% Inicia os apêndices
% ---
%\begin{apendicesenv}

% ----------------------------------------------------------
%\chapter{Nullam elementum urna vel imperdiet sodales elit ipsum pharetra ligula
%ac pretium ante justo a nulla curabitur tristique arcu eu metus}
% ----------------------------------------------------------
%\lipsum[55-57]

%\end{apendicesenv}
% ---

% ----------------------------------------------------------
% Anexos
% ----------------------------------------------------------
\cftinserthook{toc}{AAA}
% ---
% Inicia os anexos
% ---
%\anexos
%\begin{anexosenv}

% ---
%\chapter{Cras non urna sed feugiat cum sociis natoque penatibus et magnis dis
%parturient montes nascetur ridiculus mus}
% ---

%\lipsum[31]

%\end{anexosenv}

\end{document}
